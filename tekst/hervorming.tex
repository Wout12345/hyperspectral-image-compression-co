\chapter{Compressie na hervorming}
\label{hoofdstuk:hervorming}

Zoals we eerder aan het begin van sectie \ref{sec:orthogonaliteitscompressie} bespraken, nemen de factormatrices van de Tucker-decomposities van onze datasets een groot deel van de totale opslagruimte in. Voor hoog-dimensionale tensoren is dit effect echter kleiner en bestaat de compressie bijna volledig uit de kerntensor. Onze originele data is 3D, maar in dit hoofdstuk zullen we deze hervormen naar 5D en onderzoeken of we hiervoor effici\"entere compressie-algoritmes kunnen ontwikkelen.

\section{Hervormen van de datasets}

In dit hoofdstuk zullen we de originele datasets telkens omzetten van drie naar vijf modes. Dit gebeurt aan de hand van hervorming, wat simpelweg neerkomt op een herinterpretatie van het geheugen. Bijvoorbeeld, stel dat we een 1D-tensor hebben met 200 elementen. Men kan deze interpreteren als een tensor met vorm (200), maar ook als een 2D-tensor met vorm (10, 20). Als men het dan heeft over het element op positie $(i, j)$, komt dit neer op het element met index $20i + j$ (in het geval van de \textit{row-major}-conventie). Merk op dat de interne opslag van de tensor hiervoor niet aangepast moet worden, alleen de metadata.\\

Men kan deze operatie uitvoeren op een willekeurige verzameling modes van een tensor. Zo zullen we vanaf nu onze datasets hervormen door elke spatiale mode op te splitsen in twee nieuwe modes. Op deze manier zijn de afmetingen van alle modes redelijk klein, op de spectrale mode na, die sowieso al erg goed comprimeert. Ter illustratie, bij Mauna Kea komt dit neer op een hervorming van (2704, 729, 199) naar (52, 52, 27, 27, 199).\\

We zullen in dit hoofdstuk werken met datasets waarvan de spatiale modes afmetingen hebben die exact kwadraten zijn, zodat de initi\"ele mode van grootte $n^2$ kan opgesplitst worden in twee modes met grootte $n$. Op zich zijn andere factorizaties ook mogelijk, maar om praktische redenen beperken we ons in ons onderzoek tot kwadratische afmetingen. Mauna Kea en Pavia Centre voldoen al aan deze beperking. Indian Pines en Cuprite niet, dus we zullen deze datasets voor de komende hoofdstukken vervangen door versies waarbij alleen de pixels met de laagste indices uitgeknipt worden. Hierbij ronden we de afmeting van elke spatiale mode af naar het dichtsbijzijnde kleinere kwadraat.

\section{Tucker-decompositie met hervorming}

Als eerste compressiemethode kunnen we simpelweg de ST-HOSVD berekenen van de hervormde tensor. In figuur \ref{fig:reshaped_tucker_st_hosvd_results} kan men de afweging zien tussen compressiefout en -factor voor verschillende datasets, zonder (blauw) en met (oranje) het hervormen van de data. Alleen de ST-HOSVD werd uitgevoerd, zonder orthogonaliteitscompressie (dit helpt sowieso al minder in het geval met hervorming, aangezien de factormatrices kleiner zijn), quantisatie of encodering. We zien dat na deze eerste compressiefase de Tucker-decompositie even goed tot significant slechter scoort afhankelijk van de dataset en gewenste fout. Om deze reden zullen we de Tucker-decompositie met hervorming niet verder onderzoeken.

\begin{figure}[H]
\centering
\begin{subfigure}{0.48\textwidth}
  \centering
  \includegraphics[width=\linewidth]{images/reshaped_tucker_st_hosvd_results_Indian_Pines.png}
  \caption{Indian Pines}
\end{subfigure}
\begin{subfigure}{0.48\textwidth}
  \centering
  \includegraphics[width=\linewidth]{images/reshaped_tucker_st_hosvd_results_Cuprite.png}
  \caption{Cuprite}
\end{subfigure}
\\
\begin{subfigure}{0.48\textwidth}
  \centering
  \includegraphics[width=\linewidth]{images/reshaped_tucker_st_hosvd_results_Pavia_Centre.png}
  \caption{Pavia Centre}
\end{subfigure}
\begin{subfigure}{0.48\textwidth}
  \centering
  \includegraphics[width=\linewidth]{images/reshaped_tucker_st_hosvd_results_Mauna_Kea.png}
  \caption{Mauna Kea}
\end{subfigure}
\caption{Resultaten van de Tucker-decompositie zonder (blauw) en met (oranje) hervorming voor verschillende datasets, na alleen het uitvoeren van de ST-HOSVD (dus zonder orthogonaliteitscompressie, quantisatie of encodering).}
\label{fig:reshaped_tucker_st_hosvd_results}
\end{figure}

\section{Tensor trains}

Naast de Tucker-decompositie bestaan er ook andere tensordecomposities, zoals \textit{tensor trains} \cite{ref:tensor_trains}. Deze decompositie is specifiek nuttig voor het comprimeren van hoog-dimensionale tensoren, waardoor ze interessant is om te onderzoeken in dit hoofdstuk.

\subsection{De decompositie}

Ter herinnering: bij een Tucker-decompositie benaderen we een tensor $A \in \mathbb{R}^{n_1 \times n_2 \times \dots \times n_d}$ door een kerntensor $B \in \mathbb{R}^{r_1 \times r_2 \times \dots \times r_d}$ en factormatrices $U_1 \in \mathbb{R}^{n_1 \times r_1}$, $U_2 \in \mathbb{R}^{n_2 \times r_2}$, $\dots$, $U_d \in \mathbb{R}^{n_d \times r_d}$. De reconstructie gebeurt dan als volgt:
\[
A \approx B \times_1 U_1 \times_2 U_2 \times_3 \dots \times_d U_d
\]
waarbij $C \times_n U_n$ het n-mode product is van de tensor $C$ en matrix $U_k$. Een tensor train is vrij analoog, maar voegt voor elke mode een hervormingsstap toe. Hierbij hebben we een kerntensor $B \in \mathbb{R}^{r_{d-1},n_d}$, factormatrices $U_1 \in \mathbb{R}^{n_1 \times r_1}$, $U_2 \in \mathbb{R}^{r_1 n_2 \times r_2}$, $U_3 \in \mathbb{R}^{r_2 n_3 \times r_3}$, $\dots$, $U_{d-1} \in \mathbb{R}^{r_{d-2} n_{d-1} \times r_{d-1}}$ en de volgende formule:
\[
A \approx h(\dots h(h(B, r_{d-1} \times n_d) \times_1 U_{d-1}, r_{d-2} \times n_{d-1} \times n_d) \times_1 U_{d-2} \dots, r_1 \times n_2 \times \dots \times n_d) \times_1 U_1
\]
waarbij $h(C, \text{vorm})$ de hervormingsfunctie is die de tensor $C$ hervormt naar de gegeven vorm. Ter illustratie zullen we de constructieprocedure van een Tucker-decompositie vergelijken met die van een tensor train. Bij het uitvoeren van een normale ST-HOSVD op een 5D-tensor ($n_1 \times n_2 \times n_3 \times n_4 \times n_5$) ondergaat de kerntensor de volgende transformaties:

\begin{enumerate}
\item Comprimeer naar $r_1 \times n_2 \times n_3 \times n_4 \times n_5$
\item Comprimeer naar $r_1 \times r_2 \times n_3 \times n_4 \times n_5$
\item Comprimeer naar $r_1 \times r_2 \times r_3 \times n_4 \times n_5$
\item Comprimeer naar $r_1 \times r_2 \times r_3 \times r_4 \times n_5$
\item Comprimeer naar $r_1 \times r_2 \times r_3 \times r_4 \times r_5$
\end{enumerate}

Met kleine aanpassingen aan de ST-HOSVD kunnen we deze procedure even goed gebruiken om een tensor train te berekenen, waarbij de volgende stappen uitgevoerd worden:

\begin{enumerate}
\item Comprimeer naar $r_1 \times n_2 \times n_3 \times n_4 \times n_5$
\item Hervorm naar $r_1 n_2 \times n_3 \times n_4 \times n_5$
\item Comprimeer naar $r_2 \times n_3 \times n_4 \times n_5$
\item Hervorm naar $r_2 n_3 \times n_4 \times n_5$
\item Comprimeer naar $r_3 \times n_4 \times n_5$
\item Hervorm naar $r_3 n_4 \times n_5$
\item Comprimeer naar $r_4 \times n_5$
\item Hervorm naar $r_4 n_5$
\end{enumerate}

Elke iteratie, na de compressie, voegen we dus de twee eerste modes van de tensor samen, tot er slechts \'e\'en mode overblijft. Men zou kunnen redeneren dat, als men een $k$-D tensor van $n \times n \times \dots \times n$ comprimeert, telkens naar een compressierang $r$, de tensor train slechts $O(knr^2)$ ruimte inneemt. Er zijn namelijk $nr$ elementen in de eerste factormatrix, $nr^2$ elementen in elke andere factormatrix en $rn$ elementen in de uiteindelijke kerntensor. Bij de Tucker-decompositie is de ingenomen ruimte $O(r^k + knr)$, dus voor hoge $k$ zouden tensor trains het beter moeten doen. Deze redenering houdt echter geen rekening met het effect van telkens te comprimeren naar een vaste compressierang, wat een grote fout zou kunnen introduceren. Bijgevolg zullen we experimenteel moeten bepalen hoe effectief deze methode is in de praktijk.

\begin{figure}[H]
\centering
\begin{subfigure}{0.48\textwidth}
  \centering
  \includegraphics[width=\linewidth]{images/tensor_trains_st_hosvd_results_Indian_Pines.png}
  \caption{Indian Pines}
\end{subfigure}
\begin{subfigure}{0.48\textwidth}
  \centering
  \includegraphics[width=\linewidth]{images/tensor_trains_st_hosvd_results_Cuprite.png}
  \caption{Cuprite}
\end{subfigure}
\\
\begin{subfigure}{0.48\textwidth}
  \centering
  \includegraphics[width=\linewidth]{images/tensor_trains_st_hosvd_results_Pavia_Centre.png}
  \caption{Pavia Centre}
\end{subfigure}
\begin{subfigure}{0.48\textwidth}
  \centering
  \includegraphics[width=\linewidth]{images/tensor_trains_st_hosvd_results_Mauna_Kea.png}
  \caption{Mauna Kea}
\end{subfigure}
\caption{Resultaten van de Tucker-decompositie (zonder hervorming, blauw) en tensor trains (oranje) voor verschillende datasets, na alleen het uitvoeren van de ST-HOSVD (dus zonder orthogonaliteitscompressie, quantisatie of encodering).}
\label{fig:tensor_trains_st_hosvd_results}
\end{figure}

\subsection{Eerste resultaten}

Analoog aan de vorige sectie, hebben we in figuur \ref{fig:tensor_trains_st_hosvd_results} de resultaten na alleen het uitvoeren van de ST-HOSVD vergeleken tussen tensor trains en de Tucker-decompositie (zonder hervorming). Bij Indian Pines, een kleine en minder belangrijke dataset, doen beide methoden het ongeveer even goed, terwijl in alle andere gevallen tensor trains betere resultaten geven. Het is dus zeker interessant om deze techniek verder te onderzoeken.

\subsection{Verdere uitwerking}

Zoals in hoofdstuk \ref{hoofdstuk:tucker} uitgebreid beschreven werd, moeten er bij het ontwikkelen van een volledig compressie-algoritme een aantal ontwerpkeuzes gemaakt worden. Aangezien we de meeste technieken hergebruiken, zullen we dit hier slechts kort bespreken:

\begin{enumerate}
\item \textbf{Versnellen van de ST-HOSVD:}
\begin{enumerate}
\item \textbf{Modevolgorde:} Zoals eerder, de spectrale mode wordt eerst verwerkt, gevolgd door de spatiale modes in volgorde van dalende grootte.
\item \textbf{Versnellen van de SVD:} De SVD wordt berekend aan de hand van de methode met Gram-matrix, zolang deze werkelijk kleiner is dan de originele matrix. Bij de latere modes kan $n_{i+1} n_{i+2} \dots n_k$, het aantal te verwerken vectoren, kleiner zijn dan de dimensie van deze vectoren $r_{i-1} n_i$. In deze gevallen zullen we de SVD op de normale manier berekenen.
\end{enumerate}
\item \textbf{Orthogonaliteitscompressie:} Zoals eerder gebruiken we de methode met Householder-reflecties, aangezien deze een ideaal resultaat geven.

\item \textbf{Quantisatie:}
\begin{enumerate}

\item \textbf{Kerntensor:} In figuur \ref{fig:tensor-trains-core-tensor-size} zien we dat de factormatrices bijna alle ruimte innemen bij een typische tensor train. Bijgevolg zullen we de kerntensor simpelweg niet quantiseren.

\begin{figure}[]
  \centering
  \includegraphics[scale=0.7]{images/tensor_trains_core_tensor_size.png}
  \caption{Het aandeel van de kerntensor in de compressie bij tensor trains, na alleen het uitvoeren van de ST-HOSVD.}
\label{fig:tensor-trains-core-tensor-size}
\end{figure}

\item \textbf{Factormatrices:} Zoals eerder, we gebruiken gelaagde quantisatie met norm-gebaseerde bit-aantal-selectie.
\end{enumerate}
\item \textbf{Encodering:} Zoals eerder, we zullen adaptieve encodering gebruiken zonder benaderende Huffman-codes.

\begin{figure}[]
  \centering
  \includegraphics[scale=0.7]{images/filtered_sweep_points_tensor_trains_RDS.png}
  \caption{RDS-waarden van steekproefoptima en de RDS-selectiefunctie.}
\label{fig:filtered-sweep-points-tensor-trains-RDS}
\end{figure}

\begin{figure}[]
  \centering
  \includegraphics[scale=0.7]{images/filtered_sweep_points_tensor_trains_BPF.png}
  \caption{BPF-waarden van steekproefoptima en de BPF-selectiefunctie.}
\label{fig:filtered-sweep-points-tensor-trains-BPF}
\end{figure}

\item \textbf{Parameterselectie:} Aangezien de kerntensor niet meer gequantiseerd wordt, moeten alleen de RDS- en BPF-parameters afgesteld worden. Bij het verkennen van de parameterruimte beschouwen we de volgende domeinen: $RDS \in \{0.005, 0.0052, 0.0054, \dots, 0.05\}$ en $BPF \in \{16, 15, 14, \dots, 1\}$.
\begin{enumerate}

\item \textbf{Niet-adaptief:} Als we kijken naar de parameterwaarden die corresponderen met de steekproefoptima (Pareto-effici\"ente parametercombinaties), bekomen we figuren \ref{fig:filtered-sweep-points-tensor-trains-RDS} en \ref{fig:filtered-sweep-points-tensor-trains-BPF}. We kiezen de volgende selectiefuncties, op analoge wijze als in sectie \ref{sec:parameters}:
\begin{itemize}
\item $RDS = max(0.001, 0.986746*\text{kwaliteit} - 0.001199)$
\item $BPF = max(9, round(-132.505*\text{kwaliteit} + 13.048))$
\end{itemize}

\item \textbf{Adaptief:} We zullen, zoals eerder, ook een adaptieve selectieprocedure uitwerken. Hierbij wordt de RDS weer vast gekozen aan de hand van de RDS-selectiefunctie en verlagen we de BPF-waarde (startend bij de waarde gekozen door de BPF-selectiefunctie) totdat de compressiefout de kwaliteitsparameter overschrijdt. Omdat er maar \'e\'en parameter adaptief gekozen wordt in plaats van twee, is deze procedure veel simpeler dan bij Tucker-gebaseerde compressie.\\

In figuren \ref{fig:parameter_functions_results_including_adaptive_Cuprite_tensor_trains} en \ref{fig:parameter_functions_results_including_adaptive_Indian_Pines_tensor_trains} tonen we opnieuw een vergelijking van compressie met (niet-)adaptieve parameterselectie en met optimale parameters uit de steekproef van ons onderzoek. We zien dat er voor Cuprite en Indian Pines weinig verschil is tussen het adaptieve en niet-adaptieve resultaat. We hebben ook gekeken naar grotere datasets, maar daar werkte de niet-adaptieve methode ook even goed, zonder de extra rekenkost, dus we zullen niet verder werken met adaptieve parameterselectie voor tensor trains.

\end{enumerate}
\end{enumerate}

\newpage
\begin{figure}[H]
  \centering
  \includegraphics[scale=0.7]{images/parameter_functions_results_including_adaptive_Cuprite_tensor_trains.png}
  \caption{Resultaten van compressie met en zonder adaptieve parameterselectie voor Cuprite.}
  \label{fig:parameter_functions_results_including_adaptive_Cuprite_tensor_trains}
\end{figure}

\begin{figure}[H]
  \centering
  \includegraphics[scale=0.7]{images/parameter_functions_results_including_adaptive_Indian_Pines_tensor_trains.png}
  \caption{Resultaten van compressie met en zonder adaptieve parameterselectie voor Indian Pines.}
  \label{fig:parameter_functions_results_including_adaptive_Indian_Pines_tensor_trains}
\end{figure}

\newpage
