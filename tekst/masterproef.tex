\documentclass[master=cws,masteroption=ci]{kulemt}
\setup{title={Hyperspectrale afbeeldingscompressie via tensordecomposities},
  author={Wouter Baert},
  promotor={Prof.\,dr.\,ir.\ K. Meerbergen \and Dr.\ N. Vannieuwenhoven},
  assessor={Prof. dr. R. Vandebril \and Prof. dr. ir. Ph. Dutr\'e},
  assistant={Dr.\ N. Vannieuwenhoven}}
% De volgende \setup mag verwijderd worden als geen fiche gewenst is.
\setup{filingcard,
  translatedtitle={Hyperspectral image compression via tensor decompositions},
  udc=681.3,
  shortabstract={Hier komt een heel bondig abstract van hooguit 500
    woorden. \LaTeX\ commando's mogen hier gebruikt worden. Blanco lijnen
    (of het commando \texttt{\string\pa r}) zijn wel niet toegelaten!
    \endgraf \lipsum[2]}}
% Verwijder de "%" op de volgende lijn als je de kaft wil afdrukken
%\setup{coverpageonly}
% Verwijder de "%" op de volgende lijn als je enkel de eerste pagina's wil
% afdrukken en de rest bv. via Word aanmaken.
%\setup{frontpagesonly}

% Kies de fonts voor de gewone tekst, bv. Latin Modern
\setup{font=lm}

% Hier kun je dan nog andere pakketten laden of eigen definities voorzien

% Tenslotte wordt hyperref gebruikt voor pdf bestanden.
% Dit mag verwijderd worden voor de af te drukken versie.
\usepackage[pdfusetitle,colorlinks,plainpages=false]{hyperref}

% Eigen imports
\maxsecnumdepth{subsubsection}
\usepackage{float}
\usepackage{amsmath}
\usepackage{amsfonts}
\usepackage{verbatim}
\usepackage[]{algorithm2e}
\usepackage{hhline}
\usepackage{subcaption}
\usepackage{multirow}
\usepackage{xcolor}

% Listings stuff
\usepackage{listings}
\lstset{
	basicstyle=\tiny,
	keywordstyle=\color{blue},
	stringstyle=\color{orange},
	commentstyle=\color{red},
	showstringspaces=false,
	numbers=left,
	stepnumber=2,
	numbersep=5pt,
	frame=single,
	breaklines=true,
	tabsize=2,
	breakatwhitespace=true,
	morecomment=[l]{//}
}
\lstdefinestyle{Python}{language=Python}
\lstdefinestyle{C}{language=C}

%%%%%%%
% Om wat tekst te genereren wordt hier het lipsum pakket gebruikt.
% Bij een echte masterproef heb je dit natuurlijk nooit nodig!
\IfFileExists{lipsum.sty}%
 {\usepackage{lipsum}\setlipsumdefault{11-13}}%
 {\newcommand{\lipsum}[1][11-13]{\par Hier komt wat tekst: lipsum ##1.\par}}
%%%%%%%

%\includeonly{hfdst-n}
\begin{document}

\begin{preface}
  Dit is mijn dankwoord om iedereen te danken die mij bezig gehouden heeft.
  Hierbij dank ik mijn promotor, mijn begeleider en de voltallige jury.
  Ook mijn familie heeft mij erg gesteund natuurlijk.
\end{preface}

\tableofcontents*

\begin{abstract}
  In dit \texttt{abstract} environment wordt een al dan niet uitgebreide
  samenvatting van het werk gegeven. De bedoeling is wel dat dit tot
  1~bladzijde beperkt blijft.

  \lipsum[1]
\end{abstract}

% Een lijst van figuren en tabellen is optioneel
%\listoffigures
%\listoftables
% Bij een beperkt aantal figuren en tabellen gebruik je liever het volgende:
%\listoffiguresandtables
% De lijst van symbolen is eveneens optioneel.
% Deze lijst moet wel manueel aangemaakt worden, bv. als volgt:
\begin{comment}
\chapter{Lijst van afkortingen en symbolen}
\section*{Afkortingen}
\begin{flushleft}
  \renewcommand{\arraystretch}{1.1}
  \begin{tabularx}{\textwidth}{@{}p{12mm}X@{}}
    LoG   & Laplacian-of-Gaussian \\
    MSE   & Mean Square error \\
    PSNR  & Peak Signal-to-Noise ratio \\
  \end{tabularx}
\end{flushleft}
\section*{Symbolen}
\begin{flushleft}
  \renewcommand{\arraystretch}{1.1}
  \begin{tabularx}{\textwidth}{@{}p{12mm}X@{}}
    42    & ``The Answer to the Ultimate Question of Life, the Universe,
            and Everything'' volgens de \cite{h2g2} \\
    $c$   & Lichtsnelheid \\
    $E$   & Energie \\
    $m$   & Massa \\
    $\pi$ & Het getal pi \\
  \end{tabularx}
\end{flushleft}
\end{comment}

% Nu begint de eigenlijke tekst
\mainmatter

\chapter{Inleiding}
\label{hoofdstuk:inleiding}

We zijn allemaal vertrouwd met afbeeldingen. Deze bestaan uit een 2D-rooster van pixels, waarbij elke pixel typisch bestaat uit drie kleurenwaarden die de intensiteit van rood, groen en blauw licht in de pixel beschrijven. Om deze reden zeggen we ook dat een normale afbeelding drie \textit{spectrale banden} bevat. Hiermee kan men het spectrum van zichtbaar licht goed benaderen op een manier die door het menselijk zicht waargenomen wordt op quasi identieke wijze als het originele beeld.\\

Voor sommige toepassingen is het echter ook nuttig om het gemeten lichtspectrum onder te verdelen in meer dan drie banden. In dit geval spreken we van \textit{hyperspectrale} afbeeldingen. Zo voert de AVIRIS-sensor (\textit{Airborne Visible/InfraRed Imaging Spectrometer}) van NASA \cite{ref:aviris_website} metingen uit in 224 banden over het zichtbare spectrum en een deel van het infrarood-spectrum. Toepassingen hiervan bestaan onder meer uit het opvolgen van de staat van landbouwgewassen \cite{ref:tilling}, het detecteren van afwijkingen in voedsel \cite{ref:higgins} en het in kaart brengen van mineralen op basis van luchtfoto's \cite{ref:resmini}.\\

Wanneer men hyperspectrale afbeeldingen wil opslaan, stoot men echter tegen een probleem. Een typische AVIRIS-dataset bevat miljoenen pixels en kan in een ongecomprimeerd formaat tientallen gigabytes groot zijn. Het is dus erg nuttig om compressietechnieken hiervoor te ontwikkelen.\\

In figuur \ref{fig:cuprite-bands} tonen we verschillende reeksen spectrale banden van een hyperspectrale luchtfoto van Cuprite (Nevada) in de Verenigde Staten. Er zijn significante verschillen zichtbaar tussen de banden, maar de structuur blijft duidelijk hetzelfde. Op spectraal niveau is er dus erg veel redundantie en dit kunnen we benutten om dezelfde data met minder geheugen voor te stellen.

\newpage
\begin{figure}[H]
\centering
\begin{subfigure}{0.48\textwidth}
  \centering
  \includegraphics[width=0.95\linewidth]{images/cuprite_bands_0-32.png}
  \caption{Banden 0-31}
\end{subfigure}
\begin{subfigure}{0.48\textwidth}
  \centering
  \includegraphics[width=0.95\linewidth]{images/cuprite_bands_32-63.png}
  \caption{Banden 32-62}
\end{subfigure}
\\
\begin{subfigure}{0.48\textwidth}
  \centering
  \includegraphics[width=0.95\linewidth]{images/cuprite_bands_63-95.png}
  \caption{Banden 63-94}
\end{subfigure}
\begin{subfigure}{0.48\textwidth}
  \centering
  \includegraphics[width=0.95\linewidth]{images/cuprite_bands_95-127.png}
  \caption{Banden 95-126}
\end{subfigure}
\\
\begin{subfigure}{0.48\textwidth}
  \centering
  \includegraphics[width=0.95\linewidth]{images/cuprite_bands_127-158.png}
  \caption{Banden 127-157}
\end{subfigure}
\begin{subfigure}{0.48\textwidth}
  \centering
  \includegraphics[width=0.95\linewidth]{images/cuprite_bands_158-190.png}
  \caption{Banden 158-189}
\end{subfigure}
\caption{Verschillende delen van een hyperspectrale afbeelding van Cuprite (VS) \cite{ref:ehu_aviris_cuprite}. We tonen hier slechts de 190 banden die overblijven na de voorverwerking die besproken zal worden in hoofdstuk \ref{hoofdstuk:methodologie}.}
\label{fig:cuprite-bands}
\end{figure}
\newpage

Wiskundig gezien kan men de 3D data van een hyperspectrale afbeelding beschouwen als een \textit{tensor}. Daarnaast kennen we vanuit de lineaire algebra de singulierewaardenontbinding, waarmee men met een beperkte fout matrices in een verkleind formaat kan opslaan. Aangezien men met deze decompositie effici\"ent de belangrijkste spectrale signaturen in een hyperspectrale afbeelding kan opsporen, zullen we in deze thesis compressie onderzoeken via tensordecomposities die hierop gebaseerd zijn.\\

Eerst zullen we in hoofdstuk \ref{hoofdstuk:achtergrond} enkele zaken herhalen en nieuwe concepten aanhalen zodat alle vereiste voorkennis besproken is. Het gaat hier onder andere over het verschil tussen \textit{lossless} en \textit{lossy} compressie, tensoren en hun operaties, de singulierewaardenontbinding en de Tucker-decompositie.\\

In hoofdstuk \ref{hoofdstuk:methodologie} bespreken we de methodologie achter de experimenten die we zullen uitvoeren in de volgende hoofdstukken. Daarna volgt hoofdstuk \ref{hoofdstuk:tucker} over compressie gebaseerd op de Tucker-decompositie, waarin het grootste deel van ons eigen onderzoek behandeld wordt. Hierin zullen we onderzoeken hoe we de singulierewaardenontbinding in bepaalde omstandigheden sneller kunnen berekenen, gevolgd door een analyse van de volgende stappen van de compressie: orthogonaliteitscompressie, quantisatie en encodering. Op deze manier bekomen we uiteindelijk een volledig Tucker-gebaseerd compressie-algoritme. Ten slotte bespreken we ook nog technieken om goede parameters te selecteren voor dit algoritme.\\

Hierna komt hoofdstuk \ref{hoofdstuk:hervorming}, waarin we de hyperspectrale afbeeldingen hervormen naar vijf dimensies en onderzoeken of er hiervoor betere compressiemethoden bestaan. We analyseren zowel het effect van Tucker-gebaseerde compressie als compressie gebaseerd op een nieuwe decompositie: \textit{tensor trains}.\\

In hoofdstuk \ref{hoofdstuk:resultaten} zullen we de uiteindelijke resultaten van onze compressiemethoden bespreken. Hierbij vergelijken we deze methoden zowel onderling als met enkele algemene compressie-algoritmen, waarna we het hoofdstuk afsluiten met een aantal voorbeelden van gecomprimeerde hyperspectrale afbeeldingen.\\

Ten slotte eindigen we deze thesis met hoofdstuk \ref{hoofdstuk:besluit}, waarin we de conclusies van ons werk samenvatten. Hiernaast wordt er ook aandacht besteed aan mogelijke pistes voor verder onderzoek.

\chapter{Achtergrondinformatie}
\label{hoofdstuk:achtergrond}

\begin{itemize}
\item Hyperspectrale afbeeldingen
\item compressie: lossless en lossy (JPEG)
\item tensoren, n-mode product
\item Tucker, T-HOSVD
\item ST-HOSVD: algemeen, modevolgorde
\item Deflate
\item JPEG, libx264
\end{itemize}

\chapter{Methodologie}
\label{hoofdstuk:3}

We zullen zowel bij het vergelijken van de uiteindelijke algoritmes in hoofdstuk \ref{hoofdstuk:resultaten} als bij het vergelijken van verschillende technieken en parameterwaarden in hoofdstuk \ref{hoofdstuk:tucker} en hoofdstuk \ref{hoofdstuk:tensor_trains} experimenten uitvoeren om onze conclusies te staven. Om deze reden zullen we eerst alle nodige context voor deze experimenten beschrijven in dit hoofdstuk.

\section{Algemeen}
Om rekening te houden met variantie, zullen tijdsmetingen in deze tekst uitgemiddeld worden over 10 experimenten, tenzij anders vermeld. Technisch gezien zijn de resultaten van stochastische algoritmen ook niet compleet deterministisch, maar aangezien de variantie hierop vaak minimaal is zal de nood aan een grotere steekproef geval per geval behandeld worden. Daarnaast zullen tijdsmetingen ook uitgedrukt worden in CPU-tijd.

\section{Hardware}
Tenzij anders vermeld werden de tijdsmetingen in deze tekst uitgevoerd op \'e\'en core van een Intel(R) Core(TM) i7-4700HQ CPU @ 2.40GHz processor met 6GB RAM (waarvan meestal slechts een klein deel gebruikt werd weliswaar).

\section{Compressiefactor}

De compressiefactor is een belangrijke variabele die we zullen gebruiken in deze tekst. Hiermee hebben we het over de verhouding tussen de grootte van de originele, ongecomprimeerde 3D-tensor (dus bijvoorbeeld voor een $100 \times 100 \times 10$ tensor bestaande uit 16-bit integers is dit 200000 bytes) en de grootte van de gecomprimeerde versie. Dit is echter een erg simpel formaat en niet altijd de beste manier om deze data op te slaan. Men moet er dus rekening mee houden dat men zelfs door alleen simpele lossless compressie hierop uit te voeren al een compressiefactor van bv. 1.5 kan bekomen. Bijgevolg dient deze factor vooral relatief bekeken te worden, om verschillende compressietechnieken met elkaar te kunnen vergelijken.\\

Verder zal het geheugengebruik van metadata (afmetingen van tensoren, compressieparameters, ...) verwaarloosd worden, aangezien dit veel kleiner is dan de benodigde opslagruimte voor tensoren, matrices, ...

\section{Compressiefout}

Om de fouten ge\"introduceerd door de lossy compressie te vergelijken, zullen we werken met de relatieve fout, gedefinieerd als:

\[
\text{relatieve fout} = \frac{||\text{gereconstrueerde tensor} - \text{originele tensor}||_F}{||\text{originele tensor}||_F}
\]

\section{Implementatie}

De implementatie van de besproken algoritmen gebeurde in Python, waarbij het zware rekenwerk wordt uitgevoerd via library calls, voornamelijk aan de hand van NumPy/SciPy maar ook bijvoorbeeld zlib, ... Hierbij zijn ook alternatieve strategie\"en getest (zoals het al dan niet minimaliseren van transposities, het gebruik van \texttt{numpy.einsum}, ...), waarbij telkens de meest performante werd gekozen. Deze keuzes zullen niet behandeld worden in deze tekst vanwege de sterke afhankelijkheid van de gekozen programmeertaal. Desalniettemin kan het dat er nog steeds merkbare ineffici\"enties in de implementatie zijn.

\newpage
\section{Datasets}

Om een idee te krijgen van de effici\"entie van de verschillende compressietechnieken, moet men gebruik maken van echte hyperspectrale afbeeldingen. Hieronder volgen enkele voorbeelden met hun bijbehorende eigenschappen. De afbeeldingen zijn slechts een 2D-voorstelling van de volledige data, verkregen door alle frequentiekanalen bij elkaar op te tellen en deze sommen voor te stellen met grijswaarden.

\subsection{Cuprite}

\begin{figure}[H]
  \centering
  \includegraphics[scale=0.5]{images/cuprite_sum.png}
  \caption{Cuprite, VS. Bron: AVIRIS \cite{ref:cuprite}}.
  \label{fig:cuprite_sum}
\end{figure}

\textbf{Spatiale dimensies:} $512 \times 614$\\
\textbf{Spectrale dimensie:} 224 origineel, maar voor verder gebruik zijn de frequentiekanalen 0-3, 106-113, 152-168 en 219-223 verwijderd aangezien deze artefacten bevatten (banden met waarde 0, banden met veel te hoge, willekeurig verdeelde intensiteiten en naburige banden), dus de uiteindelijke spectrale dimensie is 190.

\subsection{Pavia Centre}

\begin{figure}[H]
  \centering
  \includegraphics[scale=0.4]{images/pavia_sum.png}
  \caption{Pavia Centre, Itali\"e. Bron: ROSIS \cite{ref:pavia}}.
  \label{fig:pavia_sum}
\end{figure}

\textbf{Spatiale dimensies:} Na het verwijderen van de banden zonder informatie, $1096 \times 715$ (zie de download van \cite{ref:pavia}), maar om makkelijker te kunnen reshapen zullen we werken met de $1089 \times 676$ pixels met laagste indices.\\
\textbf{Spectrale dimensie:} 102
\chapter{De Tuckerdecompositie}
\label{hoofdstuk:tucker}

In dit hoofdstuk zullen we verder bouwen op het standaard ST-HOSVD-algoritme om tot een volledig Tucker-gebaseerd compressie-algoritme te komen. Hierbij zullen we de ST-HOSVD versnellen zonder dat de compressie hieronder lijdt, methoden bekijken om orthogonale matrices verder te comprimeren, technieken vergelijken voor het quantizeren van de kerntensor en factormatrices en ten slotte alles lossless comprimeren met Deflate.

\section{Versnellen van de ST-HOSVD}

Hoewel de focus van deze thesis ligt op de afweging tussen de compressiefactor en compressiefout, is het ook nuttig om te kijken naar de compressietijd. In deze sectie zullen enkele technieken besproken worden om de ST-HOSVD te versnellen zonder de fout merkbaar te verhogen.

\subsection{Modevolgorde}

Zoals eerder besproken is het bij de ST-HOSVD van groot belang voor de performantie om de verschillende modes in de juiste volgorde te verwerken. Aangezien we werken met hyperspectrale afbeeldingen, zal de spectrale mode typisch veel beter comprimeren dan de spatiale modes. Ter illustratie: bij het uitvoeren van het standaard ST-HOSVD-algoritme met relatieve doelfout 0.025, comprimeert Cuprite van rang $512 \times 614 \times 190$ naar de volgende rang:
\begin{table}[H]
\centering
\begin{tabular}{|l|l|}
\hline
Modevolgorde & Compressierang \\ \hline
\input{data/modevolgorde.tex}
\end{tabular}
\end{table}
De spectrale dimensie is dus zeker de best comprimeerbare en zal bijgevolg vanaf nu eerst verwerkt worden, gevolgd door de spatiale dimensies (gerangschikt van groot naar klein).

\subsection{Versnellen van de SVD}

\subsubsection{Gram-matrix}

Een belangrijke eigenschap van de SVD is diens verband met de eigenwaardenontbinding. Wanneer $A = U \Sigma V^T$, dan vormen $U$ en $\Sigma^2$ namelijk respectievelijk de eigenvectoren en eigenwaarden van $A A^T$, ook wel bekend als de Gram-matrix van de rijen van $A$ \cite{ref:svd}. Als $A \in \mathbb{R}^{m \times n}$ met $m \ll n$, dan is $A A^T \in \mathbb{R}^{m \times m}$, wat leidt tot een veel snellere eigenwaardenontbinding dan wanneer men de SVD toepast op de volledige matrix. Bij deze methode moet men wel eerst een matrixvermenigvuldiging uitvoeren met complexiteit $O(m^2 n)$, wat van dezelfde orde is als het normaal berekenen van de SVD \cite{ref:svd}, maar dit kan in de praktijk nog steeds erg nuttig zijn aangezien de matrixvermenigvuldiging een relatief eenvoudige operatie is die al sterk geoptimaliseerd is in libraries als LAPACK \cite{ref:lapack}. Verder kan deze aanpak numerieke problemen opleveren omdat we eerst de invoerdata met zichzelf vermenigvuldigen en op het einde de vierkantswortels trekken van de eigenwaarden, wat de precisie verlaagt, maar in deze toepassing hoeft dit geen probleem te zijn aangezien we de SVD sterk afknotten en dit de grootste fout zal veroorzaken. Voor ons zijn fouten van deze grootte dus irrelevant.\\

Een klein nadeel van deze techniek is dat we de matrix $V$ niet meteen hebben zoals bij het berekenen van een SVD. De matricizatie van de gecomprimeerde data $X_k$ (met compressierang $k$) moet nu dus bepaald worden als $X_k := U_k^T X$ (complexiteit $O(kmn)$) in plaats van $X_k := \Sigma_k V_k$ (complexiteit $O(kn)$), wat een hogere complexiteit heeft, maar dit is slechts een matrixvermenigvuldiging dus dit kost niet zo veel tijd meer.\\

\begin{table}[H]
\centering
\begin{tabular}{|l|l|l|}
\hline
Methode & Relatieve fout & Compressietijd (s)\\ \hline
\input{data/gram-matrix.tex}
\end{tabular}
\caption{Vergelijking tussen de standaard ST-HOSVD en de methode met de Gram-matrix voor Cuprite met relatieve doelfout 0.025 (uitgemiddeld over 10 experimenten).}
\end{table}
Dit voorbeeld bevestigt dus dat deze methode een grote versnelling kan opleveren zonder enige significante fout te introduceren.

\subsubsection{Gram-matrix met QR-decompositie}

Om de eventuele fout die veroorzaakt wordt door te werken met de Gram-matrix te verkleinen, is het een gekende techniek om eerst een QR-decompositie van $A^T$ berekenen. Hoewel we eerder bespraken dat deze fout voor onze toepassing minimaal is, is deze techniek in het algemeen interessant om naar te kijken indien het niet veel extra rekentijd kost. In dit geval berekent men eerst $A^T = QR$ en gebruikt men dan de matrix $A A^T = R^T Q^T Q R = R^T R$ in plaats van $A A^T$ als invoer voor de eigenwaardenontbinding.\\

\begin{table}[H]
\centering
\begin{tabular}{|l|l|l|}
\hline
Methode & Relatieve fout & Compressietijd (s)\\ \hline
\input{data/gram-matrix-qr.tex}
\end{tabular}
\caption{Vergelijking tussen de methode met de Gram-matrix met en zonder QR-decompositie voor Cuprite met relatieve doelfout 0.025 (uitgemiddeld over 10 experimenten).}
\end{table}

Men ziet dat de QR-decompositie veel rekentijd kost en zoals eerder vermeld is de fout ge\"introduceerd door de methode met de Gram-matrix toch al verwaarloosbaar. Er is dus geen significant verschil in de relatieve fout. Bijgevolg is deze techniek afgeraden voor onze doeleinden.

\subsubsection{Gram-matrix met Lanczos-algoritme}

Het Lanczos-algoritme \cite{ref:lanczos} is een iteratief algoritme om de grootste $k$ eigenwaarden en corresponderende eigenvectoren te vinden van een symmetrische matrix. Chen en Saad \cite{ref:saad} hebben getoond dat dit algoritme aangepast kan worden om specifiek te werken met Gram-matrices:\\

\begin{algorithm}[H]
\KwData{$A, k, q_1$}
\KwResult{$q_i$'s, $\alpha_i$'s, $\beta_i$'s}
$\beta_1 := 0$\\
$q_0 := 0$\\
\For{$i = 1, ..., k$}{
$w_i := A (A^T q_i) - \beta_i q_{i-1}$\\
$\alpha_i := \langle w_i, q_i \rangle$\\
$w_i := w_i - \alpha_i q_i$\\
$w_i := w_i - \Sigma^{i-1}_{j = 1} \langle w_i q_j \rangle q_j$\\
$\beta_{i+1} := ||w_i||$\\
$q_{i+1} := w_i/\beta_{i+1}$\\
}
\end{algorithm}

De uitvoer hiervan wordt dan samengevoegd in de volgende matrices:

\[
Q = 
\begin{bmatrix}
    q_1 & q_2 & \dots & q_k
\end{bmatrix}
\]
\[
T = \begin{bmatrix}
\alpha_1 & \beta_2 & 0 & \dots & 0 \\
\beta_2 & \alpha_2 & \beta_3 & \dots & 0 \\
0 & \beta_3 & \alpha_3 & \dots & 0 \\
\vdots & \vdots & \vdots & \ddots & \vdots \\
0 & 0 & 0 & \dots & \alpha_k
\end{bmatrix}
\]

Voor elke eigenwaarde en eigenvector $\lambda, x$ van $T$ definieert men de corresponderende Ritz-waarden en Ritz-vectoren als $\lambda, Qx$. Nu geldt dat wanneer men het aantal iteraties $k$ verhoogt, de Ritz-waarden en Ritz-vectoren convergeren naar de grootste $k$ eigenwaarden en bijbehorende eigenvectoren van $A A^T$.\\

Het interessante aan deze methode is dat men slechts $O(mn)$ rekenwerk nodig heeft per iteratie (vanwege de matrix-vectorvermenigvuldiging), dus $O(kmn)$ in totaal, in plaats van de typische $O(min(m, n)mn)$ omdat men geen matrix-matrixvermenigvuldiging meer moet berekenen. Als $k < min(m, n)$ kan men hiermee dus de complexiteit verlagen.\\

Om dit te gebruiken voor onze toepassing, moeten we ons eigen stopcriterion in het algoritme verwerken, door elke iteratie na te kijken of de som van de kwadraten van de $i$ grootste singuliere waarden van $A$ (m.a.w. de som van de $i$ grootste eigenwaarden van $A A^T$) groot genoeg is. Als we dit benaderen met de som van de Ritz-waarden, kan men deze som heel simpel berekenen als $\alpha_1 + \dots + \alpha_i$ (want de som van de eigenwaarden van een matrix is het spoor van de matrix), wat gaat in constante tijd per iteratie. Op deze manier kan men effici\"ent op basis van een doelfout tijdens het algoritme de compressierang kiezen.\\

Ten slotte, eens dat de iteraties be\"eindigd zijn en we een $Q$ en $T$ hebben, berekenen we een eigenwaardenontbinding van $T$ en transformeren de eigenvectoren met $Q$. Voor het berekenen van de eigenwaardenontbinding van een symmetrische tridiagonale $k \times k$ matrix bestaan methoden met complexiteit $O(k^2)$ en er is zelfs onderzoek gedaan naar een algoritme met complexiteit $O(k \log{k})$ \cite{ref:coakley}. Onze implementatie gebruikt de gespecialiseerde functie \texttt{scipy.linalg.eigh\_tridiagonal}, waarvan we vermoeden dat de complexiteit $O(k^2)$ is, maar helaas wordt dit nergens vermeld in de documentatie \cite{ref:eigh_tridiagonal}.

\begin{table}[H]
\centering
\footnotesize
\begin{tabular}{|l|l|l|l|l|}
\hline
Methode & Relatieve fout & Compressietijd (s) & Compressierang & Compressiefactor\\ \hline
\input{data/gram-matrix-lanczos.tex}
\end{tabular}
\normalsize
\caption{Vergelijking tussen de methode met de Gram-matrix met en zonder Lanczos-algoritme voor Cuprite met relatieve doelfout 0.025 (uitgemiddeld over 10 experimenten).}
\end{table}

Blijkbaar convergeert de som van de Ritz-waarden snel genoeg zodat ons stopcriterion de doelfout goed benadert, maar de kwaliteit van de laatste basisvectoren is slechter, waardoor de compressierang significant groter gekozen moet worden voor dezelfde fout en de compressiefactor bijna gehalveerd wordt. Dit wordt ge\"illustreerd in figuur \ref{fig:lanczos-rank-comparison}. De Lanczos-methode geeft dus wel een redelijke basis maar is niet precies genoeg voor onze toepassing. Men zou kunnen proberen dit te verhelpen door meer iteraties uit te voeren dan de compressierang, maar de rekentijd ligt met het huidige aantal iteraties al significant hoger door de extra overhead. Om deze redenen zullen we deze methode verder niet gebruiken.

\begin{figure}[H]
  \centering
  \includegraphics[scale=0.7]{images/lanczos_rank_comparison.png}
  \caption{Relatieve fout voor verschillende compressierangen en methoden bij het comprimeren van de eerste mode (de spectrale mode) van Cuprite. De getallen in de legende geven aan hoeveel Lanczos-iteraties uitgevoerd werden. Men ziet dat het nuttig kan zijn om meer iteraties uit te voeren dan de uiteindelijke compressierang.}
\label{fig:lanczos-rank-comparison}
\end{figure}

\subsubsection{Randomized SVD}

Wanneer we een afgeknotte SVD zoeken van een matrix $A \in \mathbb{R}^{m \times n}$ met $m \ll n$, zijn we eigenlijk ge\"interesseerd in de belangrijkste linker-singuliere vectoren, of met andere woorden, de beste basisvectoren om de verzameling kolomvectoren van $A$ in voor te stellen. Het is echter logisch dat vaak slechts een kleine deelverzameling van de kolomvectoren van $A$ al representatief is voor de volledige populatie en tot een even goede basis leidt. Op deze manier kan men de rekentijd van de SVD verkleinen zonder een grote fout te introduceren. Deze techniek wordt de \textit{randomized SVD} \cite{ref:randomized_svd} genoemd en wordt voor veel toepassingen gebruikt om de SVD te versnellen.\\

Concreet construeren we bij deze methode dus een matrix met een willekeurige verzameling kolommen van $A$ en berekenen we van deze steekproefmatrix de SVD om zo (de eerste kolommen van) $U$ te benaderen. De $\Sigma$ die hierbij berekend wordt is weliswaar niet accuraat, aangezien deze slechts het aandeel van de singuliere vectoren in de steekproef voorstelt, maar dit kan makkelijk gecorrigeerd worden door deze te vermenigvuldigen met $\sqrt{\frac{\text{populatiegrootte}}{\text{steekproefgrootte}}}$. Hierdoor wordt $\Sigma$ ook goed benaderd en kan deze gebruikt worden om bijvoorbeeld de compressierang te bepalen.\\

Verder kan de de SVD van de steekproefmatrix ook berekend worden met de bovenstaande methoden, om zo een grotere versnelling te bekomen.\\

\begin{table}[]
\centering
\begin{tabular}{|l|l|l|}
\hline
Methode & Gram-matrix & Steekproef + Gram-matrix\\ \hhline{|=|=|=|}
\input{data/randomized-svd-cuprite-test.tex}
\end{tabular}
\caption{Vergelijking tussen een typische uitvoering van de methode met de Gram-matrix met en zonder steekproef voor Cuprite met relatieve doelfout 0.025.}
\label{table:randomized-svd-cuprite-test}
\end{table}

In tabel \ref{table:randomized-svd-cuprite-test} vindt men een typische uitvoering met en zonder steekproef. Als steekproefgrootte per mode nemen we simpelweg 5 keer de originele rang. Merk op dat dit over slechts \'e\'en experiment gaat en men dus best geen conclusies trekt over bijvoorbeeld de absolute uitvoeringstijd. Wel kan men zien dat onze benaderingsmethode voor $\Sigma$ adequate precisie geeft. Verder geeft het gebruik van een steekproef in mode 2 een significante versnelling, terwijl in de andere modes de populatie te klein was ten opzichte van de originele rang om een steekproef te nemen. Uiteindelijk is er geen significante fout ge\"introduceerd door het gebruik van de steekproef.\\

\begin{table}[]
\centering
\begin{tabular}{|l|l|l|}
\hline
Methode & Gram-matrix & Steekproef + Gram-matrix\\ \hline
\input{data/randomized-svd-cuprite-average.tex}
\end{tabular}
\caption{Vergelijking tussen de methode met de Gram-matrix met en zonder steekproef voor Cuprite met relatieve doelfout 0.025 (10 experimenten).}
\label{table:randomized-svd-cuprite-average}
\end{table}

\begin{table}[]
\centering
\begin{tabular}{|l|l|l|}
\hline
Methode & Gram-matrix & Steekproef + Gram-matrix\\ \hline
\input{data/randomized-svd-mauna-kea-average.tex}
\end{tabular}
\caption{Vergelijking tussen de methode met de Gram-matrix met en zonder steekproef voor Mauna Kea met relatieve doelfout 0.025 (10 experimenten).}
\label{table:randomized-svd-mauna-kea-average}
\end{table}

We zien in tabel \ref{table:randomized-svd-cuprite-average} dat, voor deze dataset, de randomized SVD een degelijke versnelling geeft zonder een significant verschil in de fout of compressiefactor. Daarnaast is de variantie op de fout en compressiefactor erg klein. Bovendien toont tabel \ref{table:randomized-svd-mauna-kea-average} aan dat deze techniek ook goed werkt voor voor Mauna Kea, al is het met een steekproef van 20 keer de originele rang.\\

Verder hebben we als kleine optimizatie getest of het sorteren van de steekproefindices voor het selecteren van deze kolommen uit de steekproefmatrix effect heeft op de uitvoeringstijd. Dit kost rekenwerk maar kan eventueel een versnelling geven bij het kopi\"eren van de kolommen vanwege het betere \textit{memory access pattern}. In de praktijk bleef de rekentijd ongeveer hetzelfde, dus we gebruiken deze techniek verder niet.\\

\begin{figure}[]
  \centering
  \includegraphics[scale=0.7]{images/randomized_svd_pavia_ratios.png}
  \caption{Compressiefout bij verschillende steekproefratios voor Pavia Centra met relatieve doelfout 0.025 (uitgemiddeld over 3 experimenten). De foutbalkjes stellen de minimale en maximale waarden voor over de verschillende experimenten.}
\label{fig:randomized-svd-pavia-ratios}
\end{figure}

In figuur \ref{fig:randomized-svd-pavia-ratios} ziet men echter dat de randomized SVD geen goede resultaten geeft bij Pavia Centre. Hierbij moesten we de steekproefratio (de verhouding tussen de steekproefgrootte en populatiegrootte) verhogen tot 1 om geen merkbare fout toe te voegen, maar in tabel \ref{table:randomized-svd-pavia-test} zien we dat een steekproefratio van 0.2 nog redelijk lijkt te werken voor de spectrale mode, de best comprimeerbare mode. Daarnaast zijn alle modes van Pavia significant slechter comprimeerbaar dan die van Cuprite. Hierdoor denken we dat de steekproefgrootte best groter gekozen wordt wanneer de mode slecht comprimeerbaar is.\\

De meest logische manier om de ``comprimeerbaarheid'' van een mode in te schatten, is echter door te kijken naar de verdeling van de singuliere waarden, die men pas kent na het berekenen van de SVD. Het lijkt ons dus moeilijk om voor een brede verzameling datasets op consistente wijze snel een goede steekproefgrootte te kiezen zonder verdere gegevens over de data, hoewel dit mogelijk verder onderzocht kan worden. Om deze reden zullen we de randomized SVD verder niet gebruiken, hoewel ze voor bepaalde datasets zeker een mooie versnelling kan opleveren.

\subsubsection{Besluit}

Na de resultaten van alle bovenstaande methodes te vergelijken, kiezen we ervoor om vanaf nu de methode met de Gram-matrix zonder verdere toevoegingen te gebruiken voor het berekenen van de SVD. Dit lijkt ons de snelste methode die geen significant effect heeft op de compressiefout- of factor.

\begin{table}[H]
\centering
\begin{tabular}{|l|l|l|}
\hline
Methode & Gram-matrix & Steekproef + Gram-matrix\\ \hhline{|=|=|=|}
\input{data/randomized-svd-pavia-test.tex}
\end{tabular}
\caption{Vergelijking tussen een uitvoering van de methode met de Gram-matrix met en zonder steekproef voor Pavia Centre met relatieve doelfout 0.025.}
\label{table:randomized-svd-pavia-test}
\end{table}
\section{Orthogonaliteitscompressie}

Men zou kunnen denken dat bij de Tuckerdecompositie de meeste ruimte ingenomen wordt door de kerntensor. Wanneer men namelijk naar tensoren met $k$ modes kijkt, waarbij de lengte per mode $n$ constant blijft en telkens gecomprimeerd wordt naar constante rang $r$, dan groeit de gecomprimeerde kerntensor met $O(r^k)$ en de factormatrices slechts met $O(knr)$, dus met $n, r$ constant en groeiende $k$ worden de factormatrices verwaarloosbaar.\\

In de praktijk werken we echter met een beperkt aantal modes en vaak lage compressierangen. Zelfs als we later reshapen verhogen we hiermeee $k$, maar verlagen we $n$ en $r$, dus het aandeel van de kerntensor zal hierdoor niet zozeer veel verhogen. Bijvoorbeeld, wanneer men de ST-HOSVD toepast op Cuprite met relatieve doelfout 0.025, comprimeert men van rang (512, 614, 190) naar (139, 192, 4) en nemen de factormatrices 64\% van het geheugen in. Bij Mauna Kea, een veel grotere dataset, is dit percentage 55\%. We kunnen dus concluderen dat het zeker interessant is om te kijken naar specifieke compressietechnieken voor de factormatrices.\\

We weten dat de factormatrices orthogonaal zijn en dit kunnen we benutten. Stel namelijk, we hebben een factormatrix $U \in \mathbb{R}^{n \times r}$ en verdelen deze op de volgende wijze:
\[
U = \begin{bmatrix}
A & c & \dots \\
B & x & \dots \\
\end{bmatrix}
\]
met $A \in \mathbb{R}^{(n-k) \times k}$, $B \in \mathbb{R}^{k \times k}$, $c \in \mathbb{R}^{n-k}$, $x \in \mathbb{R}^{k}$, voor willekeurige $1 \leq k < n$. Vanwege orthogonaliteit weten we dat:
\begin{align*}
\begin{bmatrix}
A \\
B \\
\end{bmatrix}^T
\begin{bmatrix}
c \\
x \\
\end{bmatrix}
&= 0 \\
A^T c + B^T x &= 0 \\
B^T x &= -A^T c
\end{align*}
Bijgevolg kunnen we $x$ berekenen als de oplossing van een lineair stelsel met $k$ onafhankelijke vergelijkingen en $k$ variabelen en moeten deze waarden niet opgeslagen worden. Theoretisch gezien kan men dus, door dit proces sequentieel uit te voeren voor $k = 1, \dots, n - 1$, een hele driehoek van $r (r - 1)/2$ elementen uit de matrix laten vallen. Om terug te komen op de eerdere voorbeelden: dit zou bij Cuprite en Mauna Kea neerkomen op 9.4\% en 7.5\% van alle waarden (inclusief kerntensor) respectievelijk. Men kan ook kiezen om de kolommen in een andere volgorde te verwerken, maar dit leek ons het beste zodat de herberekende waarden vooral zitten in de latere singuliere vectoren, die minder belangrijk zijn.
\section{Quantisatie en bitstring-compressie}

Tot nu toe hebben we altijd gewerkt met 32-bit floating-point getallen, wat voor onze doeleinden genoeg rekenprecisie geeft. Wanneer we deze getallen echter willen opslaan, zouden we ze liefst compacter willen voorstellen. Dit gaan we doen door te quantiseren: hierbij ronden we elke waarde af naar de dichtsbijzijnde waarde uit een beperkte verzameling die we zelf defini\"eren. Deze afronding zal de uiteindelijke fout op het eindresultaat verhogen, maar door deze verzamelingen goed te kiezen zullen we proberen een optimale afweging tussen de compressiefout en -factor te bekomen.\\

Hierna zullen we, als laatste fase van ons compressie-algoritme, de bits van de gequantiseerde waarden aan elkaar hangen (natuurlijk op een manier dat we bij het decoderen de waarden terug van elkaar kunnen scheiden) en lossless bitstring-compressie toepassen. Hiervoor gebruiken we het Deflate-algoritme \cite{ref:deflate} zoals ge\"implementeerd in zlib \cite{ref:zlib}. We zullen dit meteen doen na het quantiseren en alleen kijken naar de compressiefactor hierna, omdat de quantisatiemethode eventueel effect heeft op de effectiviteit van de lossless compressie.

\subsection{Kerntensor}

\subsection{Factormatrices}


\section{Afstellen van de parameters}
\chapter{Tensor trains}
\label{hoofdstuk:tensor_trains}


\chapter{Resultaten}
\label{hoofdstuk:resultaten}

Na het ontwikkelen van zowel Tucker-gebaseerde als tensor-train-gebaseerde compressie-algoritmen, is het tijd om de uiteindelijke resultaten te bekijken. In dit hoofdstuk vergelijken we onze eigen compressietechnieken zowel met elkaar, met andere algemene methoden, zoals JPEG of videocompressie, als met een methode uit de literatuur. We sluiten nog af met enkele voorbeelden van gecomprimeerde hyperspectrale afbeeldingen.

\section{Tucker versus tensor trains}

In figuren \ref{fig:tucker-vs-tensor-trains-indian-pines}, \ref{fig:tucker-vs-tensor-trains-cuprite}, \ref{fig:tucker-vs-tensor-trains-pavia-centre} en \ref{fig:tucker-vs-tensor-trains-mauna-kea} tonen we de resultaten van Tucker-gebaseerde compressie en tensor-train-gebaseerde compressie voor verschillende datasets met vari\"erende kwaliteitsparameter. Zoals eerder besproken, wordt adaptieve parameterselectie alleen gebruikt bij de Tucker-methode bij kleine datasets (Indian Pines en Cuprite). We zien dat de tensor trains het altijd minstens even goed doen als Tucker zonder adaptieve parameterselectie en dat Tucker met adaptieve parameterselectie alleen beter scoort bij Indian Pines, een kleine en minder belangrijke dataset.\\

Bovendien zien we in figuur \ref{fig:tucker-vs-tensor-trains-times} dat de methode met tensor trains minstens even snel is als niet-adaptieve Tucker en zelfs significant sneller voor grote datasets. Om deze redenen lijken tensor trains ons over het algemeen beter en zullen we hierop de focus leggen in de volgende secties.

\begin{figure}[]
  \centering
  \includegraphics[scale=0.7]{images/tucker_vs_tensor_trains_Indian_Pines.png}
  \caption{Vergelijking tussen Tucker-gebaseerde compressie en tensor-train-compressie voor Indian Pines.}
\label{fig:tucker-vs-tensor-trains-indian-pines}
\end{figure}

\begin{figure}[]
  \centering
  \includegraphics[scale=0.7]{images/tucker_vs_tensor_trains_Cuprite.png}
  \caption{Vergelijking tussen Tucker-gebaseerde compressie en tensor-train-compressie voor Cuprite.}
\label{fig:tucker-vs-tensor-trains-cuprite}
\end{figure}

\begin{figure}[]
  \centering
  \includegraphics[scale=0.7]{images/tucker_vs_tensor_trains_Pavia_Centre.png}
  \caption{Vergelijking tussen Tucker-gebaseerde compressie en tensor-train-compressie voor Pavia Centre.}
\label{fig:tucker-vs-tensor-trains-pavia-centre}
\end{figure}

\begin{figure}[]
  \centering
  \includegraphics[scale=0.7]{images/tucker_vs_tensor_trains_Mauna_Kea.png}
  \caption{Vergelijking tussen Tucker-gebaseerde compressie en tensor-train-compressie voor Mauna Kea.}
\label{fig:tucker-vs-tensor-trains-mauna-kea}
\end{figure}

\begin{figure}[]
\centering
\begin{subfigure}{0.48\textwidth}
  \centering
  \includegraphics[width=\linewidth]{images/tucker_vs_tensor_trains_times_Indian_Pines.png}
  \caption{Indian Pines}
\end{subfigure}
\begin{subfigure}{0.48\textwidth}
  \centering
  \includegraphics[width=\linewidth]{images/tucker_vs_tensor_trains_times_Cuprite.png}
  \caption{Cuprite}
\end{subfigure}
\\
\begin{subfigure}{0.48\textwidth}
  \centering
  \includegraphics[width=\linewidth]{images/tucker_vs_tensor_trains_times_Pavia_Centre.png}
  \caption{Pavia Centre}
\end{subfigure}
\begin{subfigure}{0.48\textwidth}
  \centering
  \includegraphics[width=\linewidth]{images/tucker_vs_tensor_trains_times_Mauna_Kea.png}
  \caption{Mauna Kea}
\end{subfigure}
\caption{Compressietijd in functie van relatieve fout voor Tucker-compressie en tensor trains (uitgemiddeld over 3 experimenten).}
\label{fig:tucker-vs-tensor-trains-times}
\end{figure}

\section{Vergelijking met algemene lossy compressie}

In deze sectie zullen we kort enkele algemene lossy-compressie-methoden bespreken, gevolgd door een vergelijking met onze tensor-train-gebaseerde compressie.

\newpage
\subsection{JPEG}

We hebben JPEG \cite{ref:jpeg} al aangehaald in hoofdstuk \ref{hoofdstuk:achtergrond}, maar nu leggen we uit hoe men hiermee een simpel algoritme kan maken voor het comprimeren van hyperspectrale afbeeldingen. Eerst verschuiven en schalen we de originele data, gevolgd door afronding, zodat elke waarde zich bevindt in het domein $\{0, 1, \dots, 255\}$. Deze quantisatie cre\"eert een fout die verwaarloosbaar is voor onze toepassingen. Daarna splitsen we de spectrale banden op in groepen van drie, comprimeren elke groep als \'e\'en RGB-afbeelding met JPEG en slaan de reeks gecomprimeerde afbeeldingen op. Bij de compressie wordt een kwaliteitsparameter opgegeven tussen 1 (slechtste kwaliteit) en 95 (beste kwaliteit). Deze methode houdt alleen rekening met de correlatie tussen de spectrale banden binnen elke groep, waardoor erg veel redundantie niet benut wordt en men geen goede compressie moet verwachten.

\newpage
\subsection{Videocompressie met x264}

Een betere manier om met algemene compressie de correlatie tussen de spectrale banden te gebruiken, is aan de hand van videocompressie. Een typische video bestaat namelijk uit een reeks \textit{frames} waarbij opeenvolgende frames erg op elkaar lijken, terwijl hyperspectrale afbeeldingen bestaan uit een reeks spectrale banden die allemaal qua structuur op elkaar lijken. Bovendien heeft videocompressie erg belangrijke toepassingen en wordt er al decennia met veel aandacht onderzoek gedaan in dit domein, dus men kan zich voorstellen dat moderne \textit{video codecs} erg effici\"ent zijn.\\

Concreet zullen we (zoals bij de JPEG-methode) eerst de volledige tensor quantiseren naar 8 bits, waarna we elke spectrale band doorgeven als een frame aan ffmpeg \cite{ref:ffmpeg}, een uitgebreid en veel-gebruikt programma voor het verwerken van audio en video. Voor de compressie gebruiken we de video codec x264 \cite{ref:x264}. Deze compressie gebeurt op basis van de CRF-parameter (\textit{constant rate factor}), die de afweging tussen de compressiefout en de grootte van de uitvoer bepaalt. Een CRF-waarde van 0 komt neer op lossless compressie, een waarde van 51 (het maximum) geeft de slechtste kwaliteit. Daarnaast heeft ffmpeg ook een preset-parameter, die bepaalt hoe belangrijk de compressietijd is ten opzichte van de compressiefactor en -fout. We zullen hiervoor zowel de standaardoptie \textit{medium} als de snelste optie \textit{ultrafast} proberen.

\subsection{Vergelijking}

In figuren \ref{fig:general-comparison-indian-pines}, \ref{fig:general-comparison-cuprite}, \ref{fig:general-comparison-pavia-centre} en \ref{fig:general-comparison-mauna-kea} zien we de algemene vergelijkingen op vlak van compressiefout en -factor. Zoals verwacht doet de JPEG-methode het erg slecht, dus hier zullen we verder geen rekening mee houden. Videocompressie aan de hand van x264 met preset medium scoort echter wel hoog en geeft ongeveer even goede compressie als tensor trains. Aan de andere kant zien we dat de preset ultrafast veel slechtere resultaten geeft. De resultaten van Tucker-gebaseerde compressie zijn weer even goed tot slechter dan bij tensor trains, zoals besproken in de vorige sectie, waar we ook al leerden dat deze methode trager was. Het is dus de vraag wat de verschillen in compressietijd tussen tensor trains en videocompressie dan zijn.\\

Het antwoord hierop vindt men in figuur \ref{fig:general-comparison-times}. Als het gaat om videocompressie zijn de resultaten consistent: de preset ultrafast is elke keer significant sneller dan medium. Daarnaast heeft de CRF-parameter ook slechts een beperkte invloed op de compressietijd. Dit ligt anders bij tensor trains: we merken dat de compressietijd hierbij sterker daalt in functie van de relatieve fout. Daarnaast zijn tensor trains sneller dan videocompressie voor goed comprimeerbare datasets (zie Cuprite en Mauna Kea) maar trager voor slecht comprimeerbare (zie Pavia Centre). Als men deze twee zaken combineert, kan men concluderen dat de compressietijd van onze methode met tensor trains sterk afhankelijk is van de grootte van de uitvoer van de ST-HOSVD en dat een significante hoeveelheid tijd besteed wordt aan het quantiseren en encoderen van deze uitvoer.

\newpage
\begin{figure}[H]
  \centering
  \includegraphics[scale=0.7]{images/general_comparison_new_Indian_Pines.png}
  \caption{Vergelijking tussen tensor trains en algemene compressiemethoden voor Indian Pines.}
\label{fig:general-comparison-indian-pines}
\end{figure}

\begin{figure}[H]
  \centering
  \includegraphics[scale=0.7]{images/general_comparison_new_Cuprite.png}
  \caption{Vergelijking tussen tensor trains en algemene compressiemethoden voor Cuprite.}
\label{fig:general-comparison-cuprite}
\end{figure}

\begin{figure}[H]
  \centering
  \includegraphics[scale=0.7]{images/general_comparison_new_Pavia_Centre.png}
  \caption{Vergelijking tussen tensor trains en algemene compressiemethoden voor Pavia Centre.}
\label{fig:general-comparison-pavia-centre}
\end{figure}

\begin{figure}[H]
  \centering
  \includegraphics[scale=0.7]{images/general_comparison_new_Mauna_Kea.png}
  \caption{Vergelijking tussen tensor trains en algemene compressiemethoden voor Mauna Kea.}
\label{fig:general-comparison-mauna-kea}
\end{figure}

\begin{figure}[H]
\centering
\begin{subfigure}{0.48\textwidth}
  \centering
  \includegraphics[width=\linewidth]{images/general_comparison_times_Indian_Pines.png}
  \caption{Indian Pines}
\end{subfigure}
\begin{subfigure}{0.48\textwidth}
  \centering
  \includegraphics[width=\linewidth]{images/general_comparison_times_Cuprite.png}
  \caption{Cuprite}
\end{subfigure}
\\
\begin{subfigure}{0.48\textwidth}
  \centering
  \includegraphics[width=\linewidth]{images/general_comparison_times_Pavia_Centre.png}
  \caption{Pavia Centre}
\end{subfigure}
\begin{subfigure}{0.48\textwidth}
  \centering
  \includegraphics[width=\linewidth]{images/general_comparison_times_Mauna_Kea.png}
  \caption{Mauna Kea}
\end{subfigure}
\caption{Compressietijd in functie van relatieve fout voor tensor trains en videocompressie (uitgemiddeld over 10 experimenten).}
\label{fig:general-comparison-times}
\end{figure}

Onthoud wel dat deze vergelijking op \'e\'en core gebeurde. Als men gaat paralleliseren (wat men normaal gezien doet bij videocompressie), zullen tensor trains waarschijnlijk relatief slechter scoren omdat ons algoritme eerder sequentieel ontworpen is.\\

Ten slotte besloten we om ook eens te kijken naar de decompressietijd in figuur \ref{fig:general-comparison-decompression-times}. Opnieuw is de uitvoeringstijd van tensor trains erg afhankelijk van de relatieve fout, terwijl dit bij videocompressie minder invloed heeft. Verder heeft, zoals verwacht, de preset van ffmpeg erg weinig effect op de decompressietijd; deze parameter is namelijk alleen bedoeld om de compressietijd te controleren. We concluderen dat videocompressie over het algemeen veel sneller kan decomprimeren, maar dit verschil wordt kleiner bij grote datasets (zie Mauna Kea).

\newpage
\begin{figure}[H]
\centering
\begin{subfigure}{0.48\textwidth}
  \centering
  \includegraphics[width=\linewidth]{images/general_comparison_decompression_times_Indian_Pines.png}
  \caption{Indian Pines}
\end{subfigure}
\begin{subfigure}{0.48\textwidth}
  \centering
  \includegraphics[width=\linewidth]{images/general_comparison_decompression_times_Cuprite.png}
  \caption{Cuprite}
\end{subfigure}
\\
\begin{subfigure}{0.48\textwidth}
  \centering
  \includegraphics[width=\linewidth]{images/general_comparison_decompression_times_Pavia_Centre.png}
  \caption{Pavia Centre}
\end{subfigure}
\begin{subfigure}{0.48\textwidth}
  \centering
  \includegraphics[width=\linewidth]{images/general_comparison_decompression_times_Mauna_Kea.png}
  \caption{Mauna Kea}
\end{subfigure}
\caption{Decompressietijd in functie van relatieve fout voor tensor trains en videocompressie (uitgemiddeld over 10 experimenten).}
\label{fig:general-comparison-decompression-times}
\end{figure}

\section{Vergelijking met de literatuur}

Aangezien hyperspectrale compressie een klein onderzoeksdomein is en niet altijd met dezelfde datasets gewerkt wordt, is het moeilijk om veel concrete vergelijkingspunten te vinden in de literatuur. Om deze redenen zullen we ons beperken tot \'e\'en ander resultaat: de compressie van Cuprite door Karami et al. \cite{ref:karami}, wiens paper we al eerder aanhaalden bij de literatuurstudie in hoofdstuk \ref{hoofdstuk:achtergrond}. Zij drukken de compressiefout uit als \textit{signal-to-noise ratio} (SNR) en de compressiegrootte uit als de \textit{rate} in bits per pixel per band. We zullen onze eigen resultaten omzetten met de volgende formules:
\[
SNR \text{ (in dB)} = -20 \log_{10}(\text{relatieve fout})
\]
\[
rate = \frac{\text{aantal bits per pixel per band in origineel}}{\text{compressiefactor}}
\]
In figuur \ref{fig:literature-comparison} vindt men een vergelijking van de beste eerder besproken technieken (tensor trains en x264 met preset medium) en de technieken van Karami et al. Merk op dat hun domein anders ligt dan hetgene waarvoor wij onze algoritmen afgesteld hebben (de relatieve fouten van de tensor-train-metingen gaan hier van ongeveer 0.004 tot 0.02 in plaats van 0.01 tot 0.05). Desalniettemin geven tensor trains een beter resultaat tot een SNR van 43dB (een relatieve fout van ongeveer 0.007) met een zelfs bijna verdubbelde compressiefactor voor sommige grotere fouten. Voor kleinere fouten doet de beste methode van Karami et al. het echter wel beter.

\begin{figure}[]
  \centering
  \includegraphics[scale=0.45]{images/karami_edited.png}
  \caption{Vergelijking tussen tensor trains, videocompressie met x264 (preset medium) en de compressiemethoden van Karami et al. Deze grafiek is een aangepaste versie van het origineel uit hun paper \cite{ref:karami}.}
\label{fig:literature-comparison}
\end{figure}

\newpage
\section{Voorbeeldcompressies}

Om dit hoofdstuk af te sluiten zullen we nog een aantal hyperspectrale afbeeldingen tonen voor en na compressie met tensor trains. Zoals in hoofdstuk \ref{hoofdstuk:methodologie} zullen we deze voorstellen door de spectrale banden op te tellen en deze sommen te verschuiven en schalen naar het domein $\{0, \dots, 255\}$. In deze sectie zullen relatieve fout en compressiefactor afgekort worden tot RF en CF respectievelijk.

\begin{figure}[H]
\centering
\begin{subfigure}{0.48\textwidth}
  \centering
  \includegraphics[scale=1]{images/indian_pines_cropped_sum.png}
  \caption{Origineel, 8910KB}
\end{subfigure}
\begin{subfigure}{0.48\textwidth}
  \centering
  \includegraphics[scale=1]{images/example_compression_Indian_Pines_0_01.png}
  \caption{RF: 0.0101, CF: 12.15, 733.3KB}
\end{subfigure}
\\
\begin{subfigure}{0.48\textwidth}
  \centering
  \includegraphics[scale=1]{images/example_compression_Indian_Pines_0_025.png}
  \caption{RF: 0.0231, CF: 55.89, 159.4KB}
\end{subfigure}
\begin{subfigure}{0.48\textwidth}
  \centering
  \includegraphics[scale=1]{images/example_compression_Indian_Pines_0_05.png}
  \caption{RF: 0.0475, CF: 504.75, 17.7KB}
\end{subfigure}
\caption{Voorbeeldcompressies van Indian Pines \cite{ref:ehu_aviris_indian_pines}.}
\end{figure}

\begin{figure}[]
\centering
\begin{subfigure}{\textwidth}
  \centering
  \includegraphics[scale=0.55]{images/cuprite_cropped_sum.png}
  \caption{Origineel, 103455KB}
\end{subfigure}
\\
\begin{subfigure}{\textwidth}
  \centering
  \includegraphics[scale=0.55]{images/example_compression_Cuprite_0_01.png}
  \caption{RF: 0.0098, CF: 136.2, 759.3KB}
\end{subfigure}
\end{figure}
\begin{figure}[]
\centering
\ContinuedFloat
\begin{subfigure}{\textwidth}
  \centering
  \includegraphics[scale=0.55]{images/example_compression_Cuprite_0_025.png}
  \caption{RF: 0.0261, CF: 649.6, 159.3KB}
\end{subfigure}
\begin{subfigure}{\textwidth}
  \centering
  \includegraphics[scale=0.55]{images/example_compression_Cuprite_0_05.png}
  \caption{RF: 0.0493, CF: 2888.5, 35.8KB}
\end{subfigure}
\caption{Voorbeeldcompressies van Cuprite \cite{ref:ehu_aviris_cuprite}.}
\end{figure}

\begin{figure}[]
\centering
\begin{subfigure}{\textwidth}
  \centering
  \includegraphics[width=0.85\linewidth]{images/pavia_sum.png}
  \caption{Origineel, 146657.7KB}
\end{subfigure}
\end{figure}
\begin{figure}[]
\centering
\ContinuedFloat
\begin{subfigure}{\textwidth}
  \centering
  \includegraphics[width=0.85\linewidth]{images/example_compression_Pavia_Centre_0_01.png}
  \caption{RF: 0.0138, CF: 11.67, 12569.8KB}
\end{subfigure}
\end{figure}
\begin{figure}[]
\centering
\ContinuedFloat
\begin{subfigure}{\textwidth}
  \centering
  \includegraphics[width=0.85\linewidth]{images/example_compression_Pavia_Centre_0_025.png}
  \caption{RF: 0.0298, CF: 36.8, 3985.7KB}
\end{subfigure}
\end{figure}
\begin{figure}[]
\centering
\ContinuedFloat
\begin{subfigure}{\textwidth}
  \centering
  \includegraphics[width=0.85\linewidth]{images/example_compression_Pavia_Centre_0_05.png}
  \caption{RF: 0.0502, CF: 90.7, 1617.6KB}
\end{subfigure}
\caption{Voorbeeldcompressies van Pavia Centre \cite{ref:ehu_rosis}.}
\end{figure}

\begin{figure}[]
\centering
\begin{subfigure}{0.48\textwidth}
  \centering
  \includegraphics[width=0.8\linewidth]{images/mauna_kea_sum.png}
  \caption{Origineel, 766156.2KB}
\end{subfigure}
\begin{subfigure}{0.48\textwidth}
  \centering
  \includegraphics[width=0.8\linewidth]{images/example_compression_Mauna_Kea_0_01.png}
  \caption{RF: 0.0104, CF: 121, 6332.1KB}
\end{subfigure}
\end{figure}
\begin{figure}[]
\centering
\ContinuedFloat
\begin{subfigure}{0.48\textwidth}
  \centering
  \includegraphics[width=0.8\linewidth]{images/example_compression_Mauna_Kea_0_025.png}
  \caption{RF: 0.0257, CF: 392.67, 1951.2KB}
\end{subfigure}
\begin{subfigure}{0.48\textwidth}
  \centering
  \includegraphics[width=0.8\linewidth]{images/example_compression_Mauna_Kea_0_05.png}
  \caption{RF: 0.0496, CF: 1271.31, 602.7KB}
\end{subfigure}
\caption{Voorbeeldcompressies van Mauna Kea \cite{ref:aviris}.}
\end{figure}

\chapter{Besluit}
\label{hoofdstuk:besluit}

We sluiten deze thesis af door terug te blikken op de belangrijkste conclusies van hoofdstukken \ref{hoofdstuk:tucker}, \ref{hoofdstuk:hervorming} en \ref{hoofdstuk:resultaten}. Ten slotte overlopen we ook nog even een aantal aspecten die mogelijk verder onderzocht zouden kunnen worden om onze compressie-algoritmen te verbeteren.

\section{Conclusies}

\subsection{Tucker-gebaseerde compressie}

In hoofdstuk \ref{hoofdstuk:tucker} onderzochten we de beste aanpak voor de verschillende fases van een compressie-algoritme gebaseerd op de Tucker-decompositie. Ten eerste is er de ST-HOSVD \cite{ref:st_hosvd}. Hierbij kozen we ervoor om de modes niet zozeer allemaal te verwerken in de volgorde van groot naar klein, maar eerst prioriteit te geven aan de spectrale mode vanwege diens erg goede comprimeerbaarheid. Daarnaast onderzochten we ook verschillende technieken om de SVD per mode sneller te berekenen zonder een significante fout toe te voegen. Uiteindelijk besloten we om te werken met de eigenwaardenontbinding van de Gram-matrix, hoewel ook combinaties werden overwogen met andere technieken zoals de \textit{randomized} SVD \cite{ref:randomized_svd} of een Lanczos-gebaseerde methode \cite{ref:lanczos} \cite{ref:saad}.\\

Daarna volgt orthogonaliteitscompressie: de fase waarin we proberen de redundantie in de factormatrices te benutten. We weten namelijk dat deze matrices orthogonaal zijn, dus het aantal vrijheidsgraden per matrix ligt lager dan het aantal waarden in de matrix. Eerst probeerden we onze eigen methode met stelsels te ontwikkelen, waarbij een vast deel van de factormatrix werd weggelaten en dan gereconstrueerd aan de hand van een reeks lineaire stelsels. Deze aanpak was echter traag en had grote problemen qua precisie, die deels verholpen konden worden met enkele technieken (waarvan sommige wel een negatief effect hadden op de compressiefactor). De ideale oplossing bleek echter de methode met Householder-reflecties te zijn, gebaseerd op het algoritme voor het berekenen van QR-factorizaties aan de hand van deze reflecties \cite{ref:qr_factorization_householder}. Hiermee slaagden we erin om op een effici\"ente wijze exact het aantal waarden op te slaan dat we voor ogen hadden, zonder dat we een merkbare fout toevoegen aan de factormatrices.\\

De derde fase van ons compressie-algoritme bestaat uit quantisatie: het voorstellen van floating-point waarden met beperkte discrete verzamelingen getallen. Zowel voor het quantiseren van de kerntensor als de factormatrices bleek het erg nuttig om deze op te splitsen in lagen en gebruik te maken van de inherente structuur van deze objecten. Zo bevinden veruit grootste absolute waarden in de kerntensor zich in de posities met lage indices en heeft een vaste absolute fout op elke positie een soortgelijk effect op de globale fout. Bij de factormatrices daarentegen zijn alle waarden net genormaliseerd maar dienen de eerste kolommen met hogere precisie opgeslagen te worden aangezien deze vermenigvuldigd worden met grotere getallen uit de kerntensor. Door een combinatie van dergelijke redeneringen en experimentele resultaten ontwikkelden we specifieke technieken voor het quantiseren van de kerntensor en factormatrices.\\

Ten slotte worden in de encoderingsfase alle gequantiseerde waarden omgezet naar binaire voorstellingen, aan elkaar gehangen en finaal gecomprimeerd met het Deflate-algoritme \cite{ref:deflate}. Deze encodering kan ook nog eens op verschillende manieren gebeuren: met codes van constante lengte (zoals de standaard binaire voorstelling of Gray-codes \cite{ref:graycode}) of Huffman-codes \cite{ref:huffman_coding}. Deze laatste methode bleek erg nuttig voor het comprimeren van de ge\"encodeerde data, maar helaas kostte het een significante hoeveelheid geheugen om bij elke quantisatieblok een expliciete Huffman-boom op te slaan. Als oplossing hiervoor ontwikkelden we ook ons eigen alternatief, benaderende Huffman-codes, en hoewel deze techniek kleine verbeteringen kon boeken op vlak van compressie, was dit wel met een grote kost in rekentijd, waardoor we hier geen verder gebruik van maakten. Uiteindelijk besloten we te werken met een adaptieve encoderingsmethode, waarbij elke blok ge\"encodeerd werd met een Gray-code of Huffman-code met bijbehorende boom, afhankelijk van wat het meest effici\"ent is.\\

Na het maken van deze ontwerpbeslissingen blijven er nog drie parameters over die de uiteindelijke compressiefout en -factor be\"invloeden: de relatieve doelfout in de ST-HOSVD (RDS), de bits-parameter voor de quantisatie van de kerntensor (BPK) en de bits-parameter voor de quantisatie van de factormatrices (BPF). We voerden grootschalige experimenten uit om meer te leren over de evolutie van de optimale parameters voor verschillende uiteindelijke relatieve fouten en datasets. Hoewel de optimale RDS-waarde voor een bepaalde gewenste compressiefout vrij gemakkelijk benaderd kon worden, bleek dit moeilijker voor de BPK en BPF. Hiervoor ontwikkelden we ook een adaptief algoritme, dat iteratief nieuwe compressies test terwijl deze twee laatste parameters verlaagd worden, totdat de gewenste fout bereikt wordt. Helaas vraagt deze adaptieve methode wel te veel rekentijd voor het verwerken van grote datasets.

\newpage
\subsection{Compressie na hervorming}

Men kan, in plaats van te werken met een 3D-tensor, dezelfde hyperspectrale afbeeldingen voorstellen als een 5D-tensor aan de hand van hervorming. We onderzochten of we hiervoor effici\"entere compressiemethoden konden ontwikkelen.\\

Ten eerste bekeken we kort wat het effect is van Tucker-gebaseerde compressie op een dergelijke 5D-tensor. Alleen na het uitvoeren van de eerste compressiefase, de ST-HOSVD, bleek deze methode al significant slechter dan Tucker zonder hervorming, dus we hebben de andere fases niet verder geanalyseerd.\\

Daarnaast onderzochten we ook compressie aan de hand van een nieuwe decompositie, genaamd \textit{tensor trains}, speciaal bedoeld voor het comprimeren van hoog-dimensionale tensoren. Hierbij waren de resultaten na het uitvoeren van de ST-HOSVD veelbelovend. Bijgevolg werkten we ook de andere fases van dit compressie-algoritme uit, analoog aan hoofdstuk \ref{hoofdstuk:tucker}. De meeste gebruikte technieken werden overgenomen van Tucker-gebaseerde compressie, maar enkele aanpassingen, voornamelijk aan de parameterselectie, werden doorgevoerd om rekening te houden met de verschillende eigenschappen van tensor trains ten opzichte van de Tucker-decompositie.

\subsection{Resultaten}

Als we onze Tucker-gebaseerde en tensor-train-gebaseerde compressie-algoritmen vergelijken, bleken tensor trains over het algemeen beter te scoren zowel op vlak van compressietijd en de afweging tussen compressiefactor en -fout. Wanneer we onze resultaten vergeleken met algemene lossy compressietechnieken, zagen we dat onze simpele JPEG-methode het zoals verwacht veel slechter deed, maar dat videocompressie aan de hand van x264 \cite{ref:x264} erg goede resultaten boekte, vergelijkbaar met de onze. Op vlak van compressietijd bleek videocompressie meer of minder tijd dan tensor trains nodig te hebben afhankelijk de comprimeerbaarheid van of de dataset in kwestie. Qua decompressietijd waren tensor trains echter altijd trager.\\

Daarnaast vergeleken we ook tensor-train-compressie en videocompressie met een alternatieve hyperspectrale compressiemethode van Karami et al. \cite{ref:karami}, gebaseerd op de 3D-DCT \cite{ref:dct} en SVM's \cite{ref:svm}. Onze tensor-train-methode bleek voor grotere fouten beter te werken dan deze referentie. Ten slotte toonden we ook een aantal voorbeelden van gecomprimeerde hyperspectrale afbeeldingen, die men kan terugvinden aan het einde van hoofdstuk \ref{hoofdstuk:resultaten}.

\newpage
\section{Verder onderzoek}

Om uiteindelijk aan een redelijke hoeveelheid inhoud voor deze tekst te komen, hebben we ons tijdens deze thesis vaak beperkt in de breedte of diepgang van ons onderzoek. Er zijn dus een aantal zaken die niet behandeld werden maar ons wel interessant lijken:

\begin{itemize}
\item \textbf{Keuze doelfout per mode:} In de ST-HOSVD proberen we de doelfout gelijk te verdelen over de verschillende modes. Het is echter mogelijk dat sommige modes deze fout beter kunnen benutten en dat een ongelijke verdeling van de doelfout leidt tot een grotere compressiefactor.
\item \textbf{Keuze steekproefgrootte bij de randomized SVD:} We zagen in hoofdstuk \ref{hoofdstuk:tucker} enkele mooie resultaten van de randomized SVD, maar gebruikten deze methode uiteindelijk niet omdat het ons moeilijk leek om met beperkt rekenwerk een goede steekproefgrootte te voorspellen. Indien hiervoor technieken ontwikkeld worden, zou deze methode wel degelijk gebruikt kunnen worden om de SVD (in sommige gevallen) verder te versnellen.
\item \textbf{Verbeterde quantisatie:} Het is goed dat we bij onze quantisatie onderscheid maakten tussen het effect van verschillende waarden op het eindresultaat. Men kan hier echter veel verder in gaan. Zo was onze keuze voor een pseudo-lineaire quantisatiefunctie erg simpel en kan men ook rigoreuzer analyseren hoeveel bits men best gebruikt per waarde in functie van een kwaliteitsparameter. Voor meer informatie kan men bijvoorbeeld de Wikipedia-pagina over quantisatie \cite{ref:quantization} bekijken.
\item \textbf{Verbeterde Huffman-encodering:} In het Deflate-algoritme \cite{ref:deflate} wordt, zoals bij onze adaptieve encodering, gebruik gemaakt van verschillende encoderingsstrategie\"en voor verschillende blokken data: geen speciale encodering, Huffman-encodering met een expliciet opgeslagen code en Huffman-encodering met een vaste code. Deze laatste optie zou erg nuttig kunnen zijn in onze algoritmes als de waarden per quantisatieblok over het algemeen dezelfde verdeling volgen voor verschillende blokken en datasets. Dit vereist eerst echter aangepaste quantisatie.
\item \textbf{Verbeterde parameterselectiefuncties:} Bij het automatisch afstellen van parameters hebben we de selectiefuncties beperkt tot een erg simpele vorm. Het is mogelijk dat men zonder adaptieve technieken betere parameters kan kiezen aan de hand van verbeterde selectiefuncties.
\item \textbf{Laag-niveau implementatie van quantisatie en encodering:} Zelfs met het gebruik van \texttt{bitarray} \cite{ref:bitarray} gebeurt er bij de quantisatie en encodering nog steeds erg veel in hoog-niveau Python-code. Het zou interessant zijn om te weten hoeveel men de compressie- en decompressietijd zou kunnen verbeteren door dit te implementeren in een taal als C++.
\end{itemize}

% Indien er bijlagen zijn:
\appendixpage*          % indien gewenst
\appendix
\chapter{Code algoritmen}
\label{app:algoritmen}

\section{\texttt{st\_hosvd.py}}

Ondanks de naam, bevat dit bestand code omtrent alle algoritmen besproken in hoofdstuk \ref{hoofdstuk:tucker} en meer, niet alleen de ST-HOSVD.\\

\lstinputlisting[style=Python]{../code/st_hosvd.py}

\section{\texttt{tensor\_trains.py}}

Dit bestand bevat alleen de parameterselectie voor tensor trains, de andere aspecten worden al afgehandeld in \texttt{st\_hosvd.py}.\\

\lstinputlisting[style=Python]{../code/tensor_trains.py}
\chapter{Code voorverwerker}
\label{app:voorverwerker}

Zet hier script om data te pre-processen. TODO
\chapter{Code \texttt{bitarray}}
\label{app:bitarray}

Deze bijlage beperkt zich tot slechts tot de toegevoegde stukken code uit \texttt{\_bitarray.c}.\\

\lstinputlisting[style=C,firstline=2318,lastline=2382]{../code/bitarray/bitarray/_bitarray.c}
\lstinputlisting[style=C,firstline=2489,lastline=2531]{../code/bitarray/bitarray/_bitarray.c}
\lstinputlisting[style=C,firstline=2599,lastline=2726]{../code/bitarray/bitarray/_bitarray.c}
\lstinputlisting[style=C,firstline=2845,lastline=2950]{../code/bitarray/bitarray/_bitarray.c}

\backmatter
% Na de bijlagen plaatst men nog de bibliografie.
% Je kan de  standaard "abbrv" bibliografiestijl vervangen door een andere.
\bibliographystyle{abbrv}
\bibliography{referenties}

\end{document}

%%% Local Variables: 
%%% mode: latex
%%% TeX-master: t
%%% End: 
